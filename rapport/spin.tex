%%% Packages %%%

\documentclass[french]{article}
\usepackage[margin=3cm]{geometry} 
\usepackage{fancyhdr, color, graphicx}
\usepackage[french]{babel}
\usepackage[T1]{fontenc}
\usepackage[shortlabels]{enumitem}
\usepackage{indentfirst}
\usepackage{hyperref}
\usepackage{minted}
\usepackage{titlesec}
\usepackage{pgfplots}
\usepackage{caption,subcaption}

%%% Setup and environments %%%
\setminted{
    frame=lines,
    framesep=3mm
}
\usemintedstyle{friendly}

\titleformat*{\section}{\LARGE\bfseries}
%\titleformat*{\subsection}{\large\bfseries}

\hypersetup{
    colorlinks=true,
    linkcolor=black,
    filecolor=magenta,      
    urlcolor=blue,
    pdftitle={Devoir de Model-Checking},
}

\renewcommand{\footrulewidth}{0.8pt}

\pagestyle{fancy}

%%% Header %%%

\lhead{Louka Boivin}
\rhead{Devoir de Model-checking} 
\chead{\textbf{Vérification du tri rapide}}
\lfoot{Université de Rouen}
\rfoot{\today}

%%% Main document %%%

\begin{document}

\title{\Huge Devoir de Model-Checking  \\[0.5cm]
        \bf\Huge Vérification du tri rapide avec Spin}
\author{\Large Auteur: Louka Boivin\\ \ \\}
\date{\Large \today}

\makeatletter
    \begin{titlepage}
        \begin{center}
	   { \includegraphics[width=12cm]{img/logo-université-rouen-normandie.jpg}}
	   {\ \\ \ \\}
        \vbox{}\vspace{3cm}
            {\@title }
            {\\[0.5cm]\huge Master Informatique - 1ère année}\\[3cm] 
            {\@author}
            {\Large Enseignant: M. Bedon\\ \ \\}
            {\@date\\}
        \end{center}
    \end{titlepage}
\makeatother
%not necessary but looks fancy

\addtocontents{toc}{\large}
\tableofcontents
\newpage

\section{Rappel du problème}
Le tri rapide (\emph{quicksort}) est un algorithme efficace pour trier un tableau de $n$ éléments, avec une compléxité
moyenne en $O(n\ log(n))$.

Le but de ce projet est de prouver, en utilisant un outil de model-checking, que le tri rapide
effectue bien c'est qu'il est censé faire, c'est-à-dire trier un tableau. L'outil de model-checking utilisé ici est SPIN qui permet d'implanter un modèle et de tester des propriétés sur ce modèle.

Plus précisement, il faut tout d'abord vérifier avec SPIN que le
partionnement fait bien ce qu'il est censé faire, c'est-à-dire séparé les éléments du tableau par
rapport à un pivot, à gauche les éléments inférieurs au pivot, et à droite les éléments supérieurs.
Et ensuite, vérifié qu'après un appel à quicksort le tableau est trié.\newline

L'algorithme demandé pour implanter le tri rapide est le \textbf{schéma de partition de Lomuto},
dont le pseudo-code est le suivant :
\begin{minted}{c}
// Sorts a (portion of an) array, divides it into partitions, then sorts those
algorithm quicksort(A, lo, hi) is
    // If indices are in correct order
    if lo >= 0 && hi >= 0 && lo < hi then
        // Partition array and get pivot index
        p := partition(A, lo, hi)
        // Sort the two partitions
        quicksort(A, lo, p - 1) // Left side of pivot
        quicksort(A, p + 1, hi) // Right side of pivot

// Divides array into two partitions
algorithm partition(A, lo, hi) is
    pivot := A[hi] // The pivot must be the last element
    // Pivot index
    i := lo - 1
    for j := lo to hi do
    // If the current element is less than or equal to the pivot
    if A[j] <= pivot then
        // Move the pivot index forward
        i := i + 1
        // Swap the current element with the element at the pivot
        swap A[i] with A[j]
    return i // the pivot index
\end{minted}
\newpage

\section{Implantation du modèle}
Le modèle est implanté dans le fichier \verb:quicksort.pml:. Il contient plusieurs patrons de
processus qui sont utilisés ici plutôt comme des fonctions. Ces fonctions sont les suivantes :

\begin{itemize}
    \item \verb:random_array: : génére un tableau aléatoire pour le trier,
    \item \verb:print_array: :  affiche un tableau,
    \item \verb:partition: : effectue le partitionnement du tableau selon le schéma de partition
        de Lomuto,
    \item \verb:quicksort: : effectue le tri rapide sur un tableau,
    \item \verb:check_partition: : vérifie si le partitionnement fonctionne correctement,
    \item \verb:check_sort: : vérifie si le tri fonctionne correctement.
\end{itemize}

Ensuite le corps principale du programme est décrit dans la section \verb:init:. Selon les options
passées à SPIN, le programme effectue simplement un test, ou lance les vérifications des propriétés
du modèle.

\subsection{Initialisation}
Pour utiliser les processus comme des fonctions, la plupart des fonctions prennent en paramètre un
canal de taille 0 afin de synchroniser la fin des processus (sauf les fonctions \verb:check_sort:
et \verb:check_partition: car elles sont toujours appelées en dernière).

La section \verb:init: permet de lancer les processus voulues selon les macros définis :
\begin{itemize}
    \item Dans tous les cas la première étape est de généré un tableau aléatoire de longueur N,
        et ensuite :
    \item Si la macro \verb:SHOW_SORT: est défini, le programme va afficher le tableau, le trier
        puis le ré-afficher.
    \item Si la macro \verb:VERIFY_PARTITION: est défini, le programme va lancer \verb:partition:
        entre 0 et N-1 puis \verb:check_partition:.
    \item Si la macro \verb:VERIFY_SORT: est défini, le programme va lancer \verb:quicksort:
        puis \verb:check_sort:.
\end{itemize}

\subsection{Optimisations}
Pour optimiser la vérification des propriétés sur le modèle, j'ai utilisé différentes techniques :
\begin{enumerate}
    \item Pour simplifier le code et éviter de consommer des ressources inutiles, le tableau à
        partitionner puis trier est global, tous les processus y ont accès.

    \item Comme conseillé dans l'énoncé les types de variables ont été adaptés aux éléments
        utilisés. Dès que je le pouvais, j'ai utilisé des \verb:unsigned: afin de pouvoir définir
        le nombre de bits avec lequel ils sont stockés.

    \item L'utilisation de macros pour compiler et utiliser seulement le code nécéssaire à
        l'objectif demandé.

    \item L'utilisation des fonctions définis par SPIN comme \verb:for: ou \verb:select: pour la
        simplification du code.

    \item L'option de compilation \verb:-O2: de \verb:gcc: permet de réduire le temps d'exécution
        de la vérification (d'environ 30\%). Cependant la mémoire utilisé reste la même.
\end{enumerate}
\newpage

\subsection{Vérifications}
Pour effectuer les vérifications demandées, j'ai utilisé les assertions car je trouvais que
c'était la manière la plus simple de tester. Également les propriétés ne sont pas à vérifier en
permanence sur l'état du système mais seulement après l'exécution d'un autre processus, raison
de plus pour utiliser un processus à part avec des assertions.\newline

Les fonctions \verb:check_sort: et \verb:check_partition: font donc cela, elles parcourent le
tableau et vérifient que le tableau est bien trié ou partitionné à l'aide d'assertions :

Soit $tab$ le tableau à vérifier et $p$ l'indice du pivot :
\begin{itemize}
    \item \verb:check_sort: : $\forall i \in (0, ..,  N-2)$ $\rightarrow$ \verb:assert(tab[i] <= tab[i+1]):
    \item \verb:check_partition : : $\forall i \in (0, ..,  N-1)$ :
    \begin{itemize}
        \item si $i < p$ $\rightarrow$ \verb:assert(tab[i] <= tab[pivot]):
        \item sinon $\rightarrow$ \verb:assert(tab[i] >= tab[pivot]):
    \end{itemize}
\end{itemize}

\subsection{Lancement des tests}
Pour lancer la vérification, il faut donc utiliser les macros définis précédemment de la manière
suivante :
\begin{minted}{bash}
spin -a -D{VERIFY_SORT ou VERIFY_PARTITION} quicksort.pml
gcc -O2 pan.c -o sort
./sort
\end{minted}

Et si l'on veut seulement tester le tri une fois pour un tableau généré aléatoirement :
\begin{minted}{bash}
spin -DSHOW_SORT quicksort.pml
\end{minted}

Afin d'éviter de taper ces commandes à chaque fois un makefile est disponible, dont la notice
d'utilisation est en annexe (voir section \ref{annexe:notice}).

\newpage

\section{Résultats et conclusions}
En utilisant le modèle precédemment décrit, j'ai pu vérifier que le partitionnement et le tri
fonctionnait parfaitement peu importe les valeurs du tableau, jusqu'à une taille de tableau de 7.

\subsection{Évaluation des performances}
Voici quelques graphiques montrant les performances de la vérification avec SPIN en fonction
de la taille du tableau et de la propriété à vérifier :

\begin{figure}[!h]
    \centering
    \hspace{-1,5cm}
    \begin{subfigure}[b]{0.46\textwidth}
        \begin{tikzpicture}%[scale=0.9]
        \begin{axis}[
            title={Mémoire utilisé en fonction de la taille du tableau},
            xlabel={N : taille du tableau},
            ylabel={Mémoire utilisé (Go)},
            xmin=1, xmax=8,
            ymin=0, ymax=8,
            xtick={2,3,4,5,6,7,8},
            ytick={0,1,2,3,4,5,6,7,8},
            legend pos=north west,
            ymajorgrids=true,
            grid style=dashed,
        ]

        \addplot[
            color=blue,
            mark=square,
            ]
            coordinates { % (x,y)(x,y)...
                (2,0.128)(3,0.128)(4,0.129)(5,0.140)(6,0.3)(7,3.52)(8,12)
            };
            \addlegendentry{VERIFY\_SORT}
         
        \addplot[
            color=red,
            mark=square,
            ]
            coordinates {
                (2,0.128)(3,0.128)(4,0.129)(5,0.135)(6,0.225)(7,2.44)(8,12)
            };
            \addlegendentry{VERIFY\_PARTITION}
        \end{axis}
        \end{tikzpicture}
    \end{subfigure}
    \hfill
    \begin{subfigure}[b]{0.46\textwidth}
        \begin{tikzpicture}%[scale=0.9]
        \begin{axis}[
            title={Temps d'exécution en fonction de la taille du tableau},
            xlabel={N : taille du tableau},
            ylabel={Temps d'exécution (secondes)},
            xmin=1, xmax=8,
            ymin=0, ymax=60,
            xtick={2,3,4,5,6,7,8},
            ytick={0,15,30,45,60},
            legend pos=north west,
            ymajorgrids=true,
            grid style=dashed,
        ]

        \addplot[
            color=blue,
            mark=square,
            ]
            coordinates {
                (2,0)(3,0)(4,0)(5,0.07)(6,1.17)(7,33.3)(8,180)
            };
            \addlegendentry{VERIFY\_SORT}
         
        \addplot[
            color=red,
            mark=square,
            ]
            coordinates {
                (2,0)(3,0)(4,0.01)(5,0.04)(6,0.69)(7,23)(8,180)
            };
            \addlegendentry{VERIFY\_PARTITION}
        \end{axis}
        \end{tikzpicture}
    \end{subfigure}
\end{figure}


\emph{À titre indicatif :} ces résultats ont été trouvés en exécutant plusieurs tests sur ma machine
avec 8Go de RAM, 4Go de swap et un processeur i5 d'une fréquence de 1,6GHz.\newline

On peut tirer plusieurs informations de ces graphiques :
\begin{enumerate}
    \item Le test de vérification du partitionnement est toujours légèrement moins lourd en
        ressources. Cela est sûrement dû au fait que le tri utilise l'algorithme de partitionnement
        donc l'exécution est plus longue et consomme plus de ressources.

    \item À partir d'un tableau de taille 7, la complexité de la vérification explose autant en
        mémoire qu'en temps. Pour un tableau de taille 8, je n'ai pas réussi, même apres plusieurs
        minutes en laissant swapper ma machine, spin ne s'arrête pas.
\end{enumerate}

\subsection{Difficultés}
La principale difficulté fut d'implémenter la fonction de partionnement car j'avais mal compris
le pseudo-code donné dans l'énoncé. Je pensais que la boucle $for$ s'arrêtait à $hi-1$ et donc
je pensais qu'il manquait un swap après la boucle pour que le partitionnement se fasse
correctement.\newline

Ensuite, pour faire fonctionner correctement le quicksort il fallait utiliser les canaux SPIN pour
attendre la fin de l'exécution de certains processus. J'ai eu du mal à comprendre pourquoi
l'utilisation des canaux était nécéssaires et comment les utiliser pour faire ce que je voulais.
\newline

Également j'ai eu un problème pour effectuer mes tests, au départ je passais les options de
compilation (\verb:VERIFY_{SORT,PARTITION}:) à \verb:gcc: comme nous faisions d'habitude, mais cela
ne marchait pas. En recherchant sur internet j'ai compris que SPIN passé le pré-processeur C
avant de générer son code et donc que c'était à SPIN directement qu'il fallait passer ces
paramètres.\newline

Cependant une fois ces quelques difficultés passées le reste fut assez simple, les fonctions de
vérification n'étant pas très compliqué à écrire et ensuite le reste étant de l'optimisation,
l'ajout des macros et du Makefile.

\subsection{Question subsidiaire}
Concernant cette question, j'ai essayé de passer l'exécution du quicksort en mode concurrent.

Pour cela, plutôt que de faire, dans \verb:quicksort: :
\begin{minted}{c}
// On lance le premier appel récursif
run quicksort(c, lo, p - 1); 
c?_;    // on attend la fin du premier appel
run quicksort(c, p + 1, hi); // on lance ensuite le deuxième
c?_;
\end{minted}
j'ai changé pour :
\begin{minted}{c}
// On lance les 2 appels récursifs à quicksort
run quicksort(c, lo, p - 1); 
run quicksort(c, p + 1, hi);
c?_;    // on attend la fin des deux appels ensuite
c?_;
\end{minted}

Avec cette méthode, le tri fonctionne toujours sans problème, cependant, la complexité augmente
énormément. En effet à partir d'un tableau de taille 6 ma machine ne parvient à finir la 
vérification du tri.

\begin{figure}[!h]
    \centering
    \begin{tikzpicture}
        \begin{axis}[
            title={Mémoire utilisée en fonction de la taille du tableau (avec concurrence)},
            xlabel={N : taille du tableau},
            ylabel={Mémoire utilisée (Go)},
            xmin=1, xmax=6,
            ymin=0, ymax=8,
            xtick={2,3,4,5,6},
            ytick={0,1,2,3,4,5,6,7,8},
            legend pos=north west,
            ymajorgrids=true,
            grid style=dashed,
        ]

        \addplot[
            color=blue,
            mark=square,
        ]
        coordinates {
            (2,0.128)(3,0.128)(4,0.144)(5,1.56)(6,12)
        };
        \legend{VERIFY\_SORT}
    
        \end{axis}
    \end{tikzpicture}
\end{figure}

\newpage

\section{Annexe}
\subsection{Notice d'utilisation}
\label{annexe:notice}
Un makefile permet de lancer la vérification et le test avec SPIN.\newline

Pour tester une fois le tri et afficher le résultat, il suffit de lancer :
\begin{minted}{bash}
make test
\end{minted}

Pour lancer la vérification du tri rapide :
\begin{minted}{bash}
make sort
\end{minted}

Pour lancer la vérification du partitionnement :
\begin{minted}{bash}
make partition
\end{minted}

Et pour nettoyer les fichiers de compilations créés :
\begin{minted}{bash}
make clean
\end{minted}


\end{document}
